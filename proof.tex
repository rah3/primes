\documentclass{article}
\usepackage[ruled,vlined]{algorithm2e}
\usepackage{amsmath, amssymb, amsthm}
\usepackage[margin=1in, letterpaper, includefoot]{geometry}
\usepackage{hyperref}

% Define the theorem style (optional)
\newtheorem{theorem}{Theorem}
\newtheorem{definition}{Definition}
\newtheorem{lemma}[theorem]{Lemma}    % Uses the same counter as theorems

\begin{document}
\begin{center}
    {\Huge Twin Primes Proof}
\end{center}
\thispagestyle{empty}

\section{Verification of Prime Constellations}
Here, we provide a verification of prime constellations.

% The Formal Logic Section
\subsection{Formal Definition}
A prime constellation is defined by a base $p$ and a set of offsets $\mathcal{O}$. In Lean 4, we define the property $P_{\mathcal{O}}(p)$ as:
\begin{equation*}
    P_{\mathcal{O}}(p) \iff \bigwedge_{s \in \mathcal{O}} \text{is\_prime}(p + s)
\end{equation*}

% The Algorithm Section
\subsection{Computational Search (algorithm2e)}
\begin{algorithm}[H]
    \SetKwFunction{isP}{isPrime}
    \SetKwInOut{Input}{input}\SetKwInOut{Output}{output}
    \Input{Search limit $N$, Offsets $\mathcal{O} = \{s_1, \dots, s_k\}$}
    \Output{List $L$ of valid base primes}
    \BlankLine
    $L \gets \emptyset$\;
    \For{$p \gets 2$ \KwTo $N$}{
        $valid \gets \text{true}$\;
        \For{$s \in \mathcal{O}$}{
            \If{$\neg$ \isP{$p+s$}}{
                $valid \gets \text{false}$\;
                \textbf{break}\;
            }
        }
        \If{$valid$}{
            $L \gets L \cup \{p\}$\;
        }
    }
    \caption{K-Twin Decidability Algorithm}
\end{algorithm}

% The Mathematical Case Fix
\subsection{Proof: Exclusion of Case $[0, 2, 4]$}
\begin{theorem}
    For $p > 3$, the constellation $\{p, p+2, p+4\}$ is never prime.
\end{theorem}

\begin{proof}
    Let $p \in \mathbb{N}, p > 3$. By the pigeonhole principle applied to residues modulo 3:
    \begin{itemize}
        \item If $p \equiv 0 \pmod 3$, $p$ is composite (since $p > 3$).
        \item If $p \equiv 1 \pmod 3$, then $p+2 \equiv 1+2 \equiv 0 \pmod 3$.
        \item If $p \equiv 2 \pmod 3$, then $p+4 \equiv 2+4 \equiv 0 \pmod 3$.
    \end{itemize}
    In all cases, one element is a multiple of 3. Since the elements are $>3$, at least one must be composite.
\end{proof}

\begin{definition}
A \textbf{Twin Prime} is a pair $(p, p+2)$ where both are elements of $\mathbb{P}$.
\end{definition}

\begin{theorem}
There is only one prime triplet of the form $(p, p+2, p+4)$.
\end{theorem}

\begin{proof}
By testing $p=3$, we find $\{3, 5, 7\}$, which are all prime. As proven via modular arithmetic $\pmod 3$, no other such triplets exist.
\end{proof}

\end{document}